%-*- coding: iso-latin-1 -*-

\section{QCM}
\paragraph{Question 1.} Soit $(x_1, x_2, \dots, x_n)$ un �chantillon d'une
variable al�atoire $X$ discr�te. On suppose que $X$ suit une loi param�tris�e
par $\gamma$. La vraisemblance de $(x_1, x_2, \dots, x_n)$ est donn�e par
\begin{itemize}
\item[$\square$] $\PP(X=x_1, X=x_2, \dots, X=x_n, \gamma)$
\item[$\square$] $\PP(X=x_1, X=x_2, \dots, X=x_n | \gamma)$
\item[$\square$] $\PP(\gamma | X=x_1, X=x_2, \dots, X=x_n)$
\item[$\square$] $\prod_{i=1}^n \PP(X=x_i|\gamma)$
\item[$\square$] $\prod_{i=1}^n \PP(\gamma|X=x_i)$
\end{itemize}

\paragraph{Question 2.} Soit $X$ une loi exponentielle de param�tre
$\lambda$. L'estimateur par maximum de vraisemblance de $\lambda$ est donn� par
\begin{itemize}
\item[$\square$] $L_n = n \ln(\lambda) - \lambda \sum_{i=1}^n X_i,$ o� $(X_1, X_2, \dots, X_n)$ est un �chantillon al�atoire de $X$
\item[$\square$] $\widehat{\lambda} = n \ln(\lambda) - \lambda \sum_{i=1}^n x_i,$ o� $(x_1, x_2, \dots, x_n)$ est un �chantillon al�atoire de $X$
\item[$\square$] $L_n = \frac{n}{\sum_{i=1}^n X_i},$ o� $(X_1, X_2, \dots, X_n)$ est un �chantillon al�atoire de $X$
\item[$\square$] $\widehat{\lambda} = \frac{n}{\sum_{i=1}^n x_i},$ o� $(x_1, x_2, \dots, x_n)$ est un �chantillon al�atoire de $X$.
\end{itemize}

\paragraph{Question 3. $\bigstar$} L'estimateur de Bayes est plus proche de l'esp�rance a
priori que de l'estimateur par maximum de vraisemblance quand la taille de
l'�chantillon est
\begin{itemize}
\item[$\square$] grande
\item[$\square$] petite
\item[$\square$] �a d�pend.
\end{itemize}



\section*{Solution}

{%
\noindent
\rotatebox[origin=c]{180}{%
\noindent
\begin{minipage}[t]{\linewidth}
\paragraph{Question 1.} Par d�finition (cf. �quation~\ref{eq:likelihood}),
\[
L(x_1, x_2, \dots, x_n; \gamma) = \PP(x_1, x_2, \dots, x_n | \gamma) = \prod_{i=1}^n \PP(x_i|\gamma).
\]

\paragraph{Question 2.} 
Par d�finition la vraisemblance d'un �chantillon $(x_1, x_2, \dots, x_n)$ est donn�e par
\[
  L(x_1, x_2, \dots, x_n; \lambda) = \prod_{i=1}^n \lambda e^{- \lambda x_i}
  = \lambda^n \prod_{i=1}^n e^{- \lambda x_i},
\] 
et donc sa \textit{log-vraisemblance} vaut 
\[
  \ell(x_1, x_2, \dots, x_n; \lambda) = \ln \left(\lambda^n \prod_{i=1}^n e^{- \lambda x_i}\right)
  = n \ln(\lambda) - \lambda \sum_{i=1}^n x_i.
\]
La fonction $\lambda \mapsto n \ln(\lambda) - \lambda \sum_{i=1}^n x_i$ est
concave sur $]0, +\infty[ \rightarrow \RR$ et on peut donc la maximiser en
annulant sa d�riv�e.

On obtient \textit{l'estimation par maximum de vraisemblance} de $\lambda$ suivante :
\[
  \hatmle{\lambda} = \frac{n}{\sum_{i=1}^n x_i}
\]
et, si on appelle $(X_1, X_2, \dots, X_n)$ un �chantillon al�atoire de $X$, on
obtient \textit{l'estimateur par maximum de vraisemblance} de $\lambda$ :
\[
  L_n = \frac{n}{\sum_{i=1}^n X_n}.
\]


\paragraph{Question 3.} La tendance que nous avons observ�e sur l'exemple de la
section~\ref{sec:bayes_est} (cf. � Remarque importante �) se v�rifie en g�n�ral : plus on
observe d'�chantillons, plus on s'�loigne de l'a priori pour se rapprocher d'un
estimateur issu uniquement des donn�es.
\end{minipage}%
}%

%%% Local Variables:
%%% mode: latex
%%% TeX-master: "../../sdd_2021_poly"
%%% End:

