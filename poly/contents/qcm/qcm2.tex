%-*- coding: iso-latin-1 -*-
\section{QCM}

\paragraph{Question 1.} Soit $X$ une variable al�atoire r�elle suivant une loi
de Poisson de param�tre $\lambda$. �tant donn� un �chantillon al�atoire
$(X_1, X_2, \dots, X_n)$ de $X$, et une de ses r�alisations
$(x_1, x_2, \dots, x_n)$, cocher le(s) estimateur(s) non biais�(s) de $\lambda$
parmi les propositions ci-dessous :
\begin{itemize}
\item[$\square$] $L_1 = \frac1n \sum_{i=1}^n x_i.$
\item[$\square$] $L_2 = \frac1n \sum_{i=1}^n X_i.$
\item[$\square$] $L_3 = \frac1n \sum_{i=1}^n \left( X_i^2 - \left( \frac1n \sum_{j=1}^n X_j \right)^2 \right).$
\item[$\square$] $L_4 = \frac1n \sum_{i=1}^n \left( x_i^2 - \left( \frac1n \sum_{j=1}^n x_j \right)^2 \right).$
\end{itemize}


\paragraph{Indice.}
{%
\noindent
\rotatebox[origin=c]{180}{%
\noindent
\begin{minipage}[t]{\linewidth}
Quelles sont l'esp�rance et la variance d'une loi de Poisson de param�tre $\lambda$ ?
\end{minipage}%
}%

\paragraph{Question 2.} Un estimateur biais� peut �tre plus pr�cis qu'un estimateur non-biais�.
\begin{itemize}
\item[$\square$] Vrai.
\item[$\square$] Faux.
\end{itemize}


\section*{Solution}
{%
\noindent
\rotatebox[origin=c]{180}{%
\noindent
\begin{minipage}[t]{\linewidth}
\paragraph{Question 1.} Il y a ici tout d'abord une question de vocabulaire :
un estimateur est une variable al�atoire, tandis qu'une estimation est sa
r�alisation. Ainsi nous ne consid�rons que les formules avec $X$ et non pas
avec $x$.

On rappelle que $\EE(X) = \lambda$ et $\VV(X) = \lambda.$

Seul $L_2$ est un estimateur sans biais de $\EE(X) = \lambda$ : c'est la moyenne
empirique de $X$.

On peut refaire le calcul : les $X_i$ �tant i.i.d. de m�me loi que $X,$ 
\[
  \text{B}(L_2) = \EE(L_2) - \lambda = \frac1n \sum_{i=1}^n \EE(X_i) - \lambda = 0.
\]

$L_3$ est la variance empirique de $X$ et $L_3$ est donc un estimateur
biais� de $\VV(X) = \lambda$. \newline

\paragraph{Question 2.} Vrai. C'est le concept du compromis biais-variance
(cf. section~\ref{sec:precision_estimateur}).
\end{minipage}%
}%

%%% Local Variables:
%%% mode: latex
%%% TeX-master: "../../sdd_2021_poly"
%%% End:
